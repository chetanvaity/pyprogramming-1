\documentclass{hw}
\hwname{6 / Radicals / Notes}

\begin{document}

\section*{\normalsize Class 1}

\subsection*{\normalsize Introduction}
\begin{itemize}
    \item Squares: $5 \times 5 = 5^2$
    \item Cubes: $5 \times 5 \times 5 = 5^3$
    \item Fourth powers: $5 \times 5 \times 5 \times 5 = 5^4$
\end{itemize}
Why the special name for "square"?

\subsection*{Square roots}
Math has many symbols for concepts: $\infty$ (infinity), $-$ (minus), $\pi$ (pi), etc.
\\ \smallskip
$\sqrt{}$ is the symbol for square root.
\\ \smallskip
\emph{Examples:}
\begin{itemize}
    \item $\sqrt{25} = 5$
    \item $\sqrt{64} = 8$
    \item $\sqrt{9} = 3$
\end{itemize}
The verbs are "squaring" and "taking the square root."
\\ \smallskip

\subsection*{\normalsize Perfect Squares}
\begin{center}
\begin{tabular}{|c|c|}
    \hline
    Square & Square Root \\
    \hline
    1 & 1 \\
    \hline
    4 & 2 \\
    \hline
    9 & 3 \\
    \hline
    16 & 4 \\
    \hline
    \vdots & \vdots \\
    \hline
    625 & 25 \\
    \hline
\end{tabular}
\end{center}

\subsection*{Definition}
\fbox{
If  $a^2 = b$, then $a$ is the square root of $b$.
}
Square root solves the equation $y^2 = x$.

\subsubsection*{\normalsize Principal Square Root}
It is the positive square root of $x$, denoted by $\sqrt{x}$. \\
$7^2 = 49$ \quad and \quad $(-7)^2 = 49$ \\
So, both $7$ and $-7$ are square roots of $49$. But $\sqrt{49} = 7$ (the positive).\\
The symbol $\sqrt{}$ is only used for a positive root of a positive real number.

\subsection*{\normalsize Properties}
\[
\sqrt{a \cdot b} = \sqrt{a} \cdot \sqrt{b} \\
\sqrt{\frac{a}{b}} = \frac{\sqrt{a}}{\sqrt{b}}
\]

Also note,
\[
\sqrt{a} + \sqrt{b} \neq \sqrt{a + b}.
\]
$5 = \sqrt{25} = \sqrt{16 + 9} \neq \sqrt{16} + \sqrt{9} = 4 + 3 = 7$

\emph{Proof:}
\[
\text{Let } \sqrt{a} = m, \quad \sqrt{b} = n
\]
\[
\text{therefore \em} m \cdot m = a \quad \text{and} \quad n \cdot n = b
\]
\[
a \cdot b = m \cdot m \cdot n \cdot n = (m \cdot n) \cdot (m \cdot n) = (m n)^2
\]
\[
\sqrt{a b} = m n
\]
\[
\sqrt{a b} = \sqrt{a} \cdot \sqrt{b}.
\]

\subsection*{\normalsize Simplification problems}
\textbf{Examples:}
\begin{enumerate}[label=\alph*.]
    \item Simplify $\sqrt{32}$:
        \begin{align*}
            \sqrt{32} &= \sqrt{16 \cdot 2} \\
                      &= 4\sqrt{2}
        \end{align*}
    \item Simplify $\sqrt{108}$:
        \begin{align*}
            \sqrt{108} &= \sqrt{36 \cdot 3} \\
                       &= 6\sqrt{3}
        \end{align*}
\end{enumerate}

\end{document}