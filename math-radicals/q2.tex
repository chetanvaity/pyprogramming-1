\documentclass{hw}
\hwname{6 / Radicals / q-2}

\begin{document}

\section*{Quiz 2}

\subsection*{\normalsize Instructions}
\begin{itemize}
    \item Write all steps and mention the laws or properties used.
    \item There is 1 extra point to be earned for clarity and neatness in your work.
\end{itemize}

\subsection*{\normalsize 1. Circle the simplified form (1 point each)}
\begin{enumerate}[label=\alph*.]
    \item $\sqrt{8}$
        \begin{enumerate}[label=\Alph*.]
            \item $2\sqrt{3}$
            \item $2\sqrt{2}$ % Ans
            \item $4\sqrt{2}$
            \item $16\sqrt{2}$
        \end{enumerate}
    \item $\sqrt{98}$
        \begin{enumerate}[label=\Alph*.]
            \item $2\sqrt{7}$
            \item $7\sqrt{7}$
            \item $7\sqrt{2}$ %Ans
            \item $7/2$
        \end{enumerate}
    \item $\sqrt{400}$
        \begin{enumerate}[label=\Alph*.]
            \item $2\sqrt{10}$
            \item $4\sqrt{10}$
            \item $40$
            \item $20$ % Ans
        \end{enumerate}
        \item $\sqrt[4]{x^4}$
        \begin{enumerate}[label=\Alph*.]
            \item $x$
            \item $x^4$
            \item $x^{1/4}$
            \item $1/x$
        \end{enumerate}
        \item $\frac{1}{\sqrt{50}}$
        \begin{enumerate}[label=\Alph*.]
            \item $\frac{1}{10\sqrt{5}}$
            \item $\frac{1}{5\sqrt{10}}$
            \item $\frac{1}{2\sqrt{5}}$
            \item $\frac{1}{5\sqrt{2}}$
        \end{enumerate}
        % 5 more
\end{enumerate}

\newpage
\subsection*{\normalsize 2. Complete these sentences (5 points)}
\begin{enumerate}[label=\alph*.]
    \setlength{\itemsep}{2em} % Increase space between items
    \item The ``square root of $x$'' gives the solution to the equation \blankline{6em}
    \item The ``principal square root of $x$'' is the \blankline{6em} \textit{(positive/negative/last/best)} square root of $x$
        and it is denoted by \blankline{6em} \textit{($\sqrt{x}$ / $x^2$ / $\sqrt{x^2}$ / $\frac{1}{x}$)}
    \item Consider the expression, $\dfrac{1}{3-\sqrt{2}}$. To rationalize the denominator, we multiply and divide the expression by
        $(3+\sqrt{2})$. Here, $(3+\sqrt{2})$ is the \blankline{12em} \textit{(subject/object/predicate/conjugate)} of $(3-\sqrt{2})$
    \item When we multiply something by its conjugate, we get an expression with squares like this:\\
        $(a+b)(a-b) = \blankline{10em}$
\end{enumerate}

\subsection*{\normalsize 3. Spot the Mistake}
\begin{enumerate}[label=\alph*.]
    \item A student simplified $\sqrt{72}$ as follows:
        \begin{align*}
        \sqrt{72} &= \sqrt{36 \cdot 2} \\
                    &= 36\sqrt{2}
        \end{align*}
        Identify the mistake (if any). Also, provide the correct simplification. (1 point)
        \studentxlargeworkspace
        
    \item For rationalizing the denominator, a student simplified $\dfrac{5}{\sqrt{m} + 3\sqrt{n}}$ as follows:
        \begin{align*}
            \frac{5}{\sqrt{m} - 3\sqrt{n}} &= \frac{5}{\sqrt{m} - 3\sqrt{n}} \cdot \frac{(\sqrt{m} + 3\sqrt{n})}{(\sqrt{m} + 3\sqrt{n})} \\
                                           &= \frac{5 (m + 3\sqrt{n})}{{(\sqrt{m})}^2 - {(3\sqrt{n})}^2} \\
                                           &= \frac{5m + 15\sqrt{n}}{m - 9n}
        \end{align*}
        Identify the mistake (if any). Also, provide the correct simplification. (2 points)
        \studentxxlargeworkspace
    \newpage
    \item For rationalizing the denominator, a student simplified $\dfrac{1}{x\sqrt[3]{x}}$ as follows:
        \begin{align*}
            \frac{1}{x \sqrt[3]{x}}  &= \frac{1}{x \sqrt[3]{x}} \cdot \frac{\sqrt[3]{x} \cdot \sqrt[3]{x}}{\sqrt[3]{x} \cdot \sqrt[3]{x}} \\
                                    &= \frac{\sqrt[3]{x}}{x \cdot (\sqrt[3]{x} \cdot \sqrt[3]{x} \cdot \sqrt[3]{x}) } \\
                                    &= \frac{\sqrt[3]{x}}{x \cdot x} \\
                                    &= \frac{\sqrt[3]{x}}{x^2}
        \end{align*}
        Identify the mistake (if any). Also, provide the correct simplification. (2 points)
        \studentxxlargeworkspace
    \newpage
    \item A student simplified $(12\sqrt{p} + 2)(3\sqrt{p} - 8)$ as follows:
        \begin{align*}
            (12\sqrt{p} + 2)(3\sqrt{p} - 8) &= (12 \sqrt{p} \cdot 3\sqrt{p}) - (12 \cdot 8 \sqrt{p}) + (6\sqrt{p}) - 16 \\
                                            &= (36p) - (96 \sqrt{p}) + (6\sqrt{p}) - 16 \\
                                            &= -60p + 6\sqrt{p} - 16
        \end{align*}
        Identify the mistake (if any). Also, provide the correct simplification. (2 points)
        \studentxlargeworkspace
\end{enumerate}

\newpage
\subsection*{\normalsize 4. Simplify and rationalize the denominator (2 points each)}
\begin{enumerate}[label=\alph*.]
    \item $(3 + 8\sqrt{x}) (3 - 6\sqrt{x})$
        \studentxxlargeworkspace
    \item $\dfrac{9}{\sqrt[3]{2x}}$
        \studentxxlargeworkspace
    \item $\dfrac{4}{\sqrt{7} - 6\sqrt{x}}$
        \studentxxlargeworkspace
    \item $\dfrac{12}{\sqrt[4]{x}}$
        \studentxxlargeworkspace
    \item $\dfrac{x^2 - x}{x - \sqrt{x}}$
        \studentxxlargeworkspace
\end{enumerate}

\subsection*{\normalsize 5. Compare (2 points each)}
For each problem, determine which expression is larger or if they are equal. Simplify or estimate as needed, and explain your reasoning.

\begin{enumerate}[label=\alph*.]
    \item Compare $\sqrt{300}$ and $\left( \sqrt{25} \right)^2$.
    \studentxxlargeworkspace

    \item Compare $(2x \cdot \sqrt{5x} \cdot \sqrt{20x})$ and $(10x^2 \cdot \sqrt{2})$. Assume $x > 0$.
    \studentxxlargeworkspace

    \item Compare $(2\sqrt{3} \cdot \sqrt{12})$ and $\sqrt{72}$.
    \studentxxlargeworkspace
\end{enumerate}

\newpage
\subsubsection*{\normalsize 5. Electric Force Problem Statement (3 points)}
Two charged objects are observed to attract each other with an electric force of 25 Newtons.\\
One object has a charge of 5 Coulombs, and the other object has a charge of 20 Coulombs. (Coulomb is the unit of electric charge)\\

The electric force follows the equation:
{\centering
$F = k \dfrac{q_1 q_2}{r^2}$
\par}

where,
\begin{itemize}
    \item $q_1$ and $q_2$ are the charges of the two objects in Coulombs
    \item $r$ is the distance between the objects in meters
    \item $k$, the electric constant, is equal to 3 (in this simplified universe)
    \item $F$ is the electric force between the two objects in Newtons 
\end{itemize}
\bigskip
What is the distance between the two charged objects?

\subsubsection*{\normalsize Solution}
\begin{itemize}
\item \uline{Step 1.} Substitute values in the equation:\\

{\centering
$F = k \dfrac{q_1 q_2}{r^2}$
\par}
\studentworkspace
\item \uline{Step 2.} Solve for $r$\\
\studentxlargeworkspace
\item The distance between the two charged objects is \blankline{8em} meters.
\end{itemize}

\newpage
\subsection*{\normalsize Extra Credit (4 points)}

\begin{enumerate}[label=\alph*.]
\item Simplify the expression $\dfrac{\sqrt{18x^2 \cdot \sqrt{8x^2}}}{\sqrt[4]{2}}$
(Hint: Simplify the numerator first)
\studentxxlargeworkspace
\item A square has an area of $108x^2y$ square units, where $x,y>0$. Find the side length of the square.
\studentxxlargeworkspace
\end{enumerate}

\end{document}