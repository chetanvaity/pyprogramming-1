\documentclass{hw}
\hwname{6 / Radicals / HW-4}

\begin{document}

\begin{center}
    \fbox{\parbox{\dimexpr\linewidth-2\fboxsep-2\fboxrule\relax}{
        For any nonnegative real numbers $a$ and $b$:
        \begin{align*}
        \sqrt[n]{ab} &= \sqrt[n]{a} \cdot \sqrt[n]{b} \\
        \sqrt[n]{\frac{a}{b}} &= \frac{\sqrt[n]{a}}{\sqrt[n]{b}} \quad \text{when } b > 0 \\
        \sqrt[n]{a^n} &= a
        \end{align*}
    }}
\end{center}

\section*{\normalsize 1. Circle the simplified form for the following}
\begin{enumerate}[label=\alph*.]
    \item $\sqrt{48}$
        \begin{enumerate}[label=\Alph*.]
            \item $4\sqrt{3}$
            \item $6\sqrt{2}$
            \item $3\sqrt{16}$
            \item $2\sqrt{12}$
        \end{enumerate}
        \studentworkspace
    \item $\sqrt{27}$
        \begin{enumerate}[label=\Alph*.]
            \item $2\sqrt{18}$
            \item $2\sqrt{3}$
            \item $9\sqrt{3}$
            \item $3\sqrt{3}$
        \end{enumerate}
        \studentworkspace
    \item $\sqrt{150}$
        \begin{enumerate}[label=\Alph*.]
            \item $6\sqrt{5}$
            \item $5\sqrt{6}$
            \item $25\sqrt{6}$
            \item $6\sqrt{25}$
        \end{enumerate}
        \studentworkspace
    \item $\sqrt{12}$
        \begin{enumerate}[label=\Alph*.]
            \item $3\sqrt{2}$
            \item $4\sqrt{3}$
            \item $2\sqrt{3}$
            \item $2\sqrt{6}$
        \end{enumerate}
        \studentworkspace
    \item $\sqrt{32}$
        \begin{enumerate}[label=\Alph*.]
            \item $4\sqrt{2}$
            \item $2\sqrt{4}$
            \item $2\sqrt{16}$
            \item $16\sqrt{2}$
        \end{enumerate}
        \studentworkspace
    \item $\sqrt{20}$
        \begin{enumerate}[label=\Alph*.]
            \item $10$
            \item $5\sqrt{2}$
            \item $10\sqrt{2}$
            \item $2\sqrt{5}$
        \end{enumerate}
        \studentworkspace
    \item $\sqrt{98}$
        \begin{enumerate}[label=\Alph*.]
            \item $9\sqrt{2}$
            \item $7\sqrt{2}$
            \item $2\sqrt{7}$
            \item $14$
        \end{enumerate}
        \studentworkspace
    \item $\sqrt{200}$
        \begin{enumerate}[label=\Alph*.]
            \item $10\sqrt{2}$
            \item $2\sqrt{10}$
            \item $100\sqrt{2}$
            \item $25\sqrt{8}$
        \end{enumerate}
        \studentworkspace
    \item $\sqrt{600}$
        \begin{enumerate}[label=\Alph*.]
            \item $6\sqrt{10}$
            \item $10\sqrt{6}$
            \item $6\sqrt{100}$
            \item $100\sqrt{6}$
        \end{enumerate}
        \studentworkspace
    \item $\sqrt{75}$
        \begin{enumerate}[label=\Alph*.]
            \item $3\sqrt{5}$
            \item $3\sqrt{25}$
            \item $25\sqrt{3}$
            \item $5\sqrt{3}$
        \end{enumerate}
        \studentworkspace
\end{enumerate}

\subsection*{\normalsize Simplest Radical Form}
\begin{center}
    \fbox{\parbox{\dimexpr\linewidth-2\fboxsep-2\fboxrule\relax}{
    1. No radicals appear in the denominator of a fraction.\\
    2. No fractions appear under a radical.\\
    3. All exponents in the radicand must be less than the index.\\
    4. Any exponents in the radicand can have no factors in common with the index.
}\par}
\end{center}

\textbf{Today we will focus only on the first rule.} \\


Remember that $(a+b)\cdot(a-b) = a^2 - b^2$ \\

\section*{\normalsize 2. Simplify and convert to the simplest radical form}
\begin{enumerate}[label=\alph*.]
    \item $\frac{4}{\sqrt{x}}$
    \studentxlargeworkspace
    \item $\frac{100}{\sqrt{x}}$
    \studentxlargeworkspace
    \item $\frac{2}{\sqrt{z}}$
    \studentxlargeworkspace
    \item $\frac{1}{\sqrt[3]{x}}$
    \studentxlargeworkspace
    \item $\frac{18}{\sqrt{x^5}}$
    \studentxlargeworkspace
    \item $\frac{1}{\sqrt[3]{2}}$
    \studentxlargeworkspace
    \item $\frac{7}{2\sqrt{5}}$
    \studentxlargeworkspace
    \item $\frac{1}{3 - \sqrt{x}}$
    \studentxlargeworkspace
    \item $\frac{4}{\sqrt{x} + 9}$
    \studentxlargeworkspace
    \item $\frac{1}{1-\sqrt{a}}$
    \studentxlargeworkspace
    \item $\frac{5}{4\sqrt{x} + 3}$
    \studentxxlargeworkspace
    \item $\frac{4}{\sqrt{x} + 2\sqrt{y}}$
    \studentxxlargeworkspace
    \item $\frac{3x}{\sqrt{5x} -7}$
    \studentxxlargeworkspace
\end{enumerate}

\end{document}
