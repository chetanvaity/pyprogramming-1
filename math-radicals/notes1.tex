\documentclass{hw}
\hwname{6 / Radicals / Notes}

\begin{document}

\section*{Class 1}
\subsection*{\normalsize Introduction}
\begin{itemize}
    \item Squares: $5 \times 5 = 5^2$
    \item Cubes: $5 \times 5 \times 5 = 5^3$
    \item Fourth powers: $5 \times 5 \times 5 \times 5 = 5^4$
\end{itemize}
Why the special name for "square"?

\subsection*{\normalsize Square roots}
Math has many symbols for concepts: $-$ (minus), $\infty$ (infinity), $\pi$ (pi), etc.
\\ \bigskip
$\mathbf{\sqrt{\placeholder}}$ is the symbol for square root.
\begin{itemize}
    \item $\sqrt{25} = 5$
    \item $\sqrt{64} = 8$
    \item $\sqrt{9} = 3$
\end{itemize}
The verbs are "squaring" and "taking the square root."

\subsection*{\normalsize Perfect Squares}
\begin{center}
\begin{minipage}{0.3\textwidth}
\centering
\begin{tabular}{|c|c|}
\hline
Number & Square Root \\
\hline
1 & 1 \\
4 & 2 \\
9 & 3 \\
16 & 4 \\
25 & 5 \\
36 & 6 \\
49 & 7 \\
64 & 8 \\
\hline
\end{tabular}
\end{minipage}
\begin{minipage}{0.3\textwidth}
\centering
\begin{tabular}{|c|c|}
\hline
Number & Square Root \\
\hline
81 & 9 \\
100 & 10 \\
121 & 11 \\
144 & 12 \\
169 & 13 \\
196 & 14 \\
225 & 15 \\
256 & 16 \\
\hline
\end{tabular}
\end{minipage}
\begin{minipage}{0.3\textwidth}
\centering
\begin{tabular}{|c|c|}
\hline
Number & Square Root \\
\hline
289 & 17 \\
324 & 18 \\
361 & 19 \\
400 & 20 \\
441 & 21 \\
484 & 22 \\
529 & 23 \\
576 & 24 \\
625 & 25 \\
\hline
\end{tabular}
\end{minipage}
\end{center}

\subsection*{\normalsize Definition}
\fbox{
If  $a^2 = b$, then $a$ is the square root of $b$.
}

The number inside the square-root sign is called the \textit{radicand}.

\subsubsection*{\normalsize Principal Square Root}
It is the positive square root of $x$, denoted by $\sqrt{x}$. \\
$7^2 = 49$ \quad and \quad $(-7)^2 = 49$ \\
So, both $7$ and $-7$ are square roots of $49$. But $\sqrt{49} = 7$ (the positive).\\
The symbol $\sqrt{\placeholder}$ is only used for a positive root of a real number.

\subsection*{\normalsize Properties}
{\centering
\fbox{
$\sqrt{a \cdot b} = \sqrt{a} \cdot \sqrt{b} \qquad ; \qquad \sqrt{\frac{a}{b}} = \frac{\sqrt{a}}{\sqrt{b}}$
}\par}

Let $\sqrt{a} = m, \quad \sqrt{b} = n$ \\
$\therefore m \cdot m = a \quad \text{and} \quad n \cdot n = b $
\begin{align*}
a \cdot b &= m \cdot m \cdot n \cdot n \\
          &= (m \cdot n) \cdot (m \cdot n) \\
          &= (m n)^2 \\
\sqrt{a b} &= m n \\
\sqrt{a b} &= \sqrt{a} \cdot \sqrt{b}.
\end{align*}

Also note,
$\sqrt{a} + \sqrt{b} \neq \sqrt{a + b}$ \\
$5 = \sqrt{25} = \sqrt{16 + 9} \neq \sqrt{16} + \sqrt{9} = 4 + 3 = 7$

\subsection*{\normalsize Simplification problems}
\begin{minipage}{0.5\textwidth}
\centering
    \begin{align*}
        \sqrt{32} &= \sqrt{16 \cdot 2} \\
                    &= 4\sqrt{2}
    \end{align*}
    \begin{align*}
        \sqrt{18x^2} &= \sqrt{18 \cdot x^2} \\
                     &= \sqrt{9} \cdot \sqrt{2} \cdot \sqrt{x^2} \\
                     &= 3 \cdot \sqrt{2} \cdot x \\
                     &= 3x\sqrt{2}
        \end{align*}

\end{minipage}
\begin{minipage}{0.5\textwidth}
\centering
    \begin{align*}
        \sqrt{150}  &= \sqrt{25 \cdot 6} \\
                    &= 5\sqrt{6}
    \end{align*}
    \begin{align*}
        \sqrt{\frac{36x^2}{9}} &= \frac{\sqrt{36x^2}}{\sqrt{9}} \\
                        &= \frac{\sqrt{36} \cdot \sqrt{x^2}}{\sqrt{9}} \\
                        &= \frac{6x}{3} \\
                        &= 2x
    \end{align*}
\end{minipage}

\subsection*{\normalsize Higher roots}
$\sqrt[3]{8} = 2, \text{because } 2 \cdot 2 \cdot 2 = 8$ \\
$\sqrt[4]{81} = 3, \text{because } 3 \cdot 3 \cdot 3 \cdot 3 = 81$ \\

$\text{The properties apply.}$\\
$\text{eg: }\sqrt[n]{ab} = \sqrt[n]{a}\sqrt[n]{b}$

\end{document}