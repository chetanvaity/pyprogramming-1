\documentclass{hw}
\hwname{7 / ExpLog / Notes}

\begin{document}

\subsection*{\normalsize Introduction}
Exponents and Logarithms appear in all fields - Science, Engineering, Biology, Finance, Music, etc. \\

A lot of math is about systematic notation and strict use of definitions and language.\\
The notation in this topic - especially for logarithm is a bit new. Just be careful and after some practice, it will not be so bad.\\

$a + a + a + a + a = a \times 5$ \\
$a \times a \times a \times a \times a = a^5$ \\

Just like multiplication is a shortcut to denote multiple additions, exponentiation is a shortcut notation to denote multiple multiplications.\\

Terms: Base, Exponent (Power) \\
Common bases - 2, 10, e

\subsection*{\normalsize Exponential growth/decay}
\begin{itemize}
\item Bacteria growth - $2 \rightarrow 4 \rightarrow 8 \rightarrow 16 ...$
\item Carbon dating - In the atmosphere, carbon-12 is around 99\%. And 1\% is C-14 - which is radioactive with a half-life of 5730 years.
So, if we determine the ratio of C-14/C-12 in an old wood/bone sample, we can estimate its age.
This is especially useful because all of history is within the last 10000 years.
\end{itemize}

\subsection*{\normalsize Laws of Exponents}
\begin{enumerate}[label=\roman*)]
\item $x^ax^b = x^{a+b}$
\item $\frac{x^a}{x^b} = x^{a-b} \quad \faint{; x \neq 0}$
\item $(x^a)^b = x^{ab}$
\item $(x^a y^b)^c = x^{ac} y^{bc}$
\item $x^{-a} = \frac{1}{x^a} \quad \faint{; x \neq 0}$
\item $x^{a/b} = \sqrt[b]{x^a} \quad \faint{; b \neq 0}$
\item $x^0 = 1 \quad \faint{; x \neq 0}$
\item $0^0 = Undefined$
\end{enumerate}

\subsection*{\normalsize Exercises}
\begin{itemize}
\item $9^2, \quad 9^{1/2}, \quad 9^{-1}, \quad 9^{-1/2}$
\item $27^{1/3}, \quad 32^{1/5}, \quad 100^{1/2}$
\item $27^{2/3}$
\end{itemize}

\subsection*{\normalsize Exercises}
\begin{itemize}
    \item $27^\frac{1}{3} = \sqrt[3]{27} = 3$
    \item $27^\frac{2}{3} = \sqrt[3]{27^2} = \sqrt[3]{3^6} = 3^\frac{6}{3} = 3^2 = 9$
    \item $-32^\frac{3}{5} = -\frac{1}{32^\frac{3}{5}} = -\frac{1}{(2^{5})^\frac{3}{5}} = -\frac{1}{2^3} = -\frac{1}{8}$
    \item $-9^\frac{3}{2} = -(3^2)^\frac{3}{2} = -3^3 = -27$
    \item $(27^4)^{-\frac{1}{12}} = ((3^3)^4)^{-\frac{1}{12}} = (3^{12})^{-\frac{1}{12}} = \frac{1}{(3^{12})^{\frac{1}{12}}} = \frac{1}{3}$
    \item $\sqrt[3]{p^4q} = p^\frac{4}{3} \cdot q^\frac{1}{3}$
    \item $\sqrt[3]{8b^6c^{-4}} = 2b^{2}c^{-\frac{4}{3}}$
    \item $\sqrt{8} \cdot \sqrt[6]{8} = \sqrt{2^3} \cdot \sqrt[6]{2^3} = 2^\frac{3}{2} \cdot 2^\frac{3}{6} = 2^\frac{12}{6} = 4$
    \item $\frac{\sqrt[5]{27^3}}{\sqrt[5]{9^2}} = \frac{\sqrt[5]{3^9}}{\sqrt[5]{3^4}} = \frac{3^\frac{9}{5}}{3^\frac{4}{5}} = 3^{\frac{9}{5} - \frac{4}{5} = 3}$
\end{itemize}

%% For later Word problem - Extra Credit
%
% Rocket launch competition
% Odyssey Team & Trojan Team
% Height of rocket h(t) = P^t
% P = power constant of the rocket-engine
% 
% Odyssey have built and tested a rocket which goes up 8m in 3 seconds.
% The Trojans just tested one which goes up 25m in 3 seconds.
%
% The Odyssey team is not feeling so good :-(
%
% Meanwhile, the Odyssey engine team has worked overtime to build another more powerful engine and they
% are confident that the new design will give 50% more power. This means the power constant will be 1.5 times the current value.
% 
% The team would like to know if its worth the effort to rebuild the rocket with the new engine.
% Is there a chance that the newer stronger engine will propel the rocket higher than 25m?
% ---
% Answer:
% Older rocket has P=2 (this needs be worked out by students from 8 = P^3)
% Newer rocket has P= 2*1.5 = 3
% New height = 3^3 = 27
% Its worth it!

\end{document}