\documentclass{hw}
\hwname{7 / ExpLog / q1}

\begin{document}

\section*{Quiz 1}

\subsection*{Instructions}
\begin{itemize}
    \item Write all steps and mention the laws or properties used
    \item There is 1 extra point to be earned for clarity and neatness in your work
\end{itemize}

\subsection*{1. Solve (2 points each)}
\begin{enumerate}[label=\alph*.]
    \item $x^{-\frac{1}{2}} = 6$
    \studentxxlargeworkspace
    \item $(3x-1)^{-\frac{2}{3}} = \frac{1}{4}$
    \studentxxlargeworkspace
    \item $4x + 5 = 125^{\frac{2}{3}}$
    \studentxxlargeworkspace
    \item $\frac{x^\frac{5}{2}}{x^{\frac{1}{2}}} - \frac{5x^{\frac{3}{2}}}{x^\frac{1}{2}} = -6$
    \studentxxlargeworkspace
\end{enumerate}

\subsection*{2. Solve (2 points each)}
\begin{enumerate}[label=\alph*.]
    \item $2^{2x} + 2^x - 6 = 0$
    \studentxxlargeworkspace
    \item $4^{1-x} = 8$
    \studentxxlargeworkspace
    \newpage
    \item $49^{x-2} = 7\sqrt{7}$
    \studentxxlargeworkspace
    \item $25^x - 5^x = 0$
    \studentxlargeworkspace
\end{enumerate}

\newpage
\subsection*{3. Complete these sentences (1 point each)}
\begin{enumerate}[label=\alph*.]
    \item Consider the expression $p^q$. Here $q$, is the index or power, and $p$ is \dottedline
    \item The \textbf{domain} of a function is \dottedline \dottedline
    \item For $f(x) = 2^x$, the domain is \dottedline \dottedline
    \item The \textbf{range} of a function is \dottedline \dottedline
    \item For $f(x) = 2^x$, the range is \dottedline \dottedline
    \item A function has an inverse function if and only if every horizontal line \dottedline \dottedline
\end{enumerate}

\newpage
\subsection*{4. Find inverse functions (2 points each)}
\begin{enumerate}[label=\alph*.]
    \item $y = x^3 + 2$
    \studentxxlargeworkspace
    \item $y = 3x - 7$
    \studentxxlargeworkspace
    \newpage
    \item $y = \sqrt[3]{2x}$
    \studentxxlargeworkspace
    \item $3x^5 - 9$
    \studentxxlargeworkspace
\end{enumerate}

\newpage
\uline{\textit{Extra Credit}}
(Attempt only after finishing earlier sections)

\subsubsection*{Rocket Launch (4 points)}

Two teams, Odyssey and Trojan, are competing in a rocket launch competition, where a rocket's elevation at 3 seconds after launch is the winning criteria.\\

The height of a rocket, $h$, in meters, is given by the equation:

{\centering
$ h = P^t $
\par}

where:
\begin{itemize}
    \item $P$ is the \textit{power constant} of the rocket engine
    \item $t$ is the time in seconds after launch
\end{itemize}

The Odyssey team have tested a rocket that reaches a height of 8 meters in 3 seconds. The Trojan team tested a rocket that reaches 25 meters in 3 seconds.\\
The Odyssey team is not feeling so good. \\

Meanwhile, the Odyssey engine team has worked overtime to make a new engine that provides 50\% more power. This means the new power constant is 1.5 times the current value.\\

The team would like to know if its worth the effort to rebuild the rocket with the new engine. Is there a chance that the newer stronger engine will propel the rocket higher than 25m?

\subsubsection*{Solution}
\begin{itemize}
\item The height reached by the existing rocket in 3 seconds, $h = \blankline{4em}$\\
\vspace*{0.5cm}\\
\item Determine the current rocket's power constant $P$, by substituting values in the equation $h=P^t$ \\
\vspace*{5cm}\\
Current rocket power constant, $P = \blankline{6em}$\\
\item The new rocket's power constant, $ P_{\text{new}} = \blankline{12em} $ \\
\vspace*{0.5cm}\\
\item Determine the height $h_{\text{new}}$ after 3 seconds with the new stronger engine\\
\vspace*{5cm}\\
$ h_{\text{new}} = \blankline{6em} meters.$
\vspace*{0.5cm}
\item Hence, it \blankline{8em} \textit{(is / is not)} worthwhile to rebuild the rocket with the new engine. \\
\end{itemize}

\end{document}
